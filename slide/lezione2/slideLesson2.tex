\begin{frame}
	\setbeamercolor{block body}{bg = yellow}
	\begin{block}{}
		\begin{center}
			{\large\textbf{Corso di base JAVA}}\\
			\itshape{Mauro Donadeo}\\
			mail: mauro.donadeo@gmail.com
		\end{center}
	\end{block}
	\setbeamercolor{block body}{bg = white}
	\begin{block}{}	
		\begin{center}
			\large{Scrivere i nostri primi programmi}\\
			\includegraphics[width = 30mm]{images/java-logo.jpg}
		\end{center}
	\end{block}	
\end{frame}

\begin{frame}
\frametitle{Introduzione}
\begin{block}{Cosa tratteremo}
In questa parte tenteremo di andare un po' più a fondo scrivendo diversi programmi e tenteremo di capire
alcune parti specifiche del Java. Alcune cose sono specifiche del linguaggio Java, ma la maggior parte 
sono comuni a tutti i linguaggi di programmazione.
\end{block}
\end{frame}

\section*{Variabili}
\subsection*{Assegnamento}
\begin{frame}[fragile]
\begin{block}{}
Supponiamo di avere un conto corrente con all'interno 50.00 euro. E che riceveremo un acconto per un lavoro di 500.00 euro.
\end{block}
\pause
\begin{lstlisting}
amountCount = 50.00;
amountCount = amountCount + 500.00;
\end{lstlisting}
\begin{block}{}
Nel pezzo di codice sopra viene fatto uso di \texttt{amountCount} che è una \textit{variabile}. Una variabile quindi 
serve per mantenere in memoria qualcosa. Nel nostro esempio mettiamo infatti il numero 50.00. Questo numero potrà essere
cambiato, naturalmente il vecchio valore non verrà memorizzato.
\end{block}
\end{frame}

\subsection*{Tipi di variabile}
\begin{frame}
\begin{block}{}
Quando si pensa ad un computer che memorizza una lettera ad esempio la lettera J esso in realtà memorizza la sequenza 
\texttt{01001010}, ogni cosa all'interno di un computer è una sequenza di 0 e 1, più comunemente sequenze di \textCl{bit}.
\end{block}
\begin{block}{01001010}
Questa sequenza inoltre può assumere altri significati:
\begin{itemize}
\item Come detto precedentemente la lettera J
\item ma anche il numero intero 74;
\item $1.036960863003646 x 10^{-43}$
\end{itemize}
\end{block}
\begin{block}{}
\textit{Il computer distingue il tipo della sequenza utilizzando il concetto di \textCl{tipo}}. Il tipo di una variabile è
il range di valori che può assumere.
\end{block}
\end{frame}

\begin{frame}[fragile]
\begin{lstlisting}
import static java.lang.System.out;
class acconto {
   public static void main(String args[]) {
       double amountAccount;
       amountAccount = 50.00;
       amountAccount = amountAccount + 500.00;
       out.print("You have:");
       out.print(amountInAccount);
       out.println(" in your account.");
   }	
}
\end{lstlisting}
\begin{block}{}
\texttt{double ammountAccount;}
Indica la \textit{dichiarazione di una variabile}. Nella dichiarazione della variabile la parola \texttt{double} è 
una \textit{keyword} Java.
\end{block}
\end{frame}

\subsection*{I tipi primitivi di Java}
\begin{frame}
\begin{block}{}
La parola \texttt{double} è un esempio di tipo primitivo in Java (anche come conosciuti \textit{tipi semplici}).
\end{block}
\begin{tabular}{c c c}
\toprule
\multicolumn{3}{c}{Tipi primitivi di Java}\\
\cmidrule(r){1-3}
\textbf{Tipo} & \textbf{Che valore rappresentano} & \textbf{Range di valori}\\
\hline
\texttt{byte} & \texttt{(byte)42} & -128 a 127 \\
\hline
\texttt{short} & \texttt{(short)42} & -32768 a 32767 \\
\hline
\texttt{int} & \texttt{42} & –2147483648 a \\ 
 & & 2147483647\\
\hline
\texttt{long} & \texttt{42L} & –9223372036854775808 a\\
& & 9223372036854775807\\
\bottomrule
\end{tabular}
\end{frame}

