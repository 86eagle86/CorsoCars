\begin{frame}
	\setbeamercolor{block body}{bg = yellow}
	\begin{block}{}
		\begin{center}
			{\large\textbf{Corso di base JAVA}}\\
			\itshape{Mauro Donadeo}\\
			mail: mauro.donadeo@gmail.com
		\end{center}
	\end{block}
	\setbeamercolor{block body}{bg = white}
	\begin{block}{}	
		\begin{center}
			\large{INTRODUZIONE}
			\includegraphics[width = 30mm]{images/java-logo.jpg}
		\end{center}
	\end{block}	
\end{frame}

\section{Informazioni generali}
\subsection{Corso}
\begin{frame}
\frametitle{Informazioni generali}
\begin{block}{Orari lezioni}
\begin{itemize}
\item Martedì 27 marzo {\itshape ore 13:30 - 18:30};
\item Mercoledì 28 marzo {\itshape ore 14:00 - 19:00};
\item Venerdì 30 marzo {\itshape 9:30 - 13:30};
\item Martedì 3 aprile {\itshape 13:30 - 18:30};
\item Giovedì 5 aprile {\itshape 13:30 - 18:30};
\end{itemize}
\end{block}
\begin{block}{Dove}
Tutte le lezioni si svogeranno all'interno dell'aula CAR.
\end{block}
\end{frame}

\subsection{About me}
\begin{frame}
\frametitle{About me}
\begin{block}{}
Laureato in ingengeria informatica ad ottobre del 2011. Titolo della
tesi: \textit{Realizzazionde di un sistema di video conferenza 3D utilizzando 
il sistema di video conferenza MS Kinect}.
\end{block}

\begin{block}{Posizione attuale}
\begin{itemize}
\item Collaboro con il Prof. Gamberini all'interno di HTLab (htlab.psy.unipd.it);
\item Gesture recognition su dispositivi touchless all'inteno del progetto europeo
CEEDs (ceeds-project.eu);
\item Misure di accuracy di più Kinect che funzionano contemporaneamente. Sempre con 
il fine di riconoscere gesture.
\end{itemize}
\end{block}
\end{frame}

\section{Cos'è un programma}
\begin{frame}
\frametitle{Cos'è un programma}
\begin{block}{Il computer}
Tutti sappiamo che un computer è una macchina che: 
\begin{itemize}
\item \textbf{memoriza dati} (numeri, parole, immagini suoni...);
\item \textbf{interagisce con dispositivi} (schermo, tastiere, mouse, kinect...)
\item \textbf{esegue programmi}.
\end{itemize}
\end{block}
\begin{block}{I programmi}
I programmi sono \textbf{sequenze} di \textit{istruzioni} che il \textbf{computer}
\textit{esegue}, e di \textit{decisioni} che il \textbf{computer} prende per svolgere
una certa attività.
\end{block}
\end{frame}
\subsection{Cos'è un programma}
\begin{frame}
\begin{block}{}
Nonostante i programmi sono molto sofisticati e svolgano funzioni molto complesse, le 
istruzioni di cui sono composti sono \textbf{molto elementari} per esempio:
\begin{itemize}
\item estrarre un numero da una posizione di memoria;
\item inviare un documento in stampa;
\item accendere un punto rosso in una pos. determinata dello schermo;
\item se un numero è negativo, allora si svolge una funzione più tosto che un altra.
\end{itemize}
\end{block}
\begin{block}{Programmazione}
Un programma descrive al computer in estremo dettaglio la sequenza necessaria di 
passi per svolgere un particolare compito:
\begin{center}
\itshape{L'attività di progettare e \textbf{realizzare un programma} è detta \alert{programmazione}}
\end{center}
\end{block}
\end{frame}

\subsection{Cos'è un algoritmo}
\begin{frame}
\begin{block}{Problemi}
Quale dei seguenti due problemi può essere risolto da un computer:
\begin{itemize}
\item Dato un insieme di fotografie di paesaggi, qual'è il più \textCl{rilassante}?
\item Avete un deposito di ventimila euro in un conto bancario che produce il 5%
di interessi all'anno, capitalizzati annualmente, quanti anni occorrono affinché 
il saldo del conto arrivi al doppio della cifra iniziale?
\end{itemize}
\end{block}
\pause
\begin{block}{}
Il primo problema non può essere risolto dal computer. \textbf{\textCl{Perché}}?
\end{block}
\end{frame}

\begin{frame}
\begin{block}{}
\begin{itemize}
\item \textbf{\textit{Un computer può risolvere soltanto problemi che potrebbero essere risolti anche manualmente}}:
\begin{itemize}
\item \textCl{E' solo molto più veloce, non si annoia, e non fa errori (se programmato nella maniera giusta)}
\end{itemize}
\end{itemize}
\end{block}
\pause
\begin{block}{Cos'è un algoritmo}
Si dice \textCl{algoritmo} la \alert{descrizione} di un metodo di soluzione di un problema che:
\begin{itemize}
\item sia eseguibile;
\item sia priva di ambiguità;
\item arrivi ad una conclusione in un tempo finito.
\end{itemize}
\end{block}
\pause
\begin{block}{}
\textit{Un computer può risolvere soltanto quei problemi per i quali sia noto un algoritmo}
\end{block}
\end{frame}

\begin{frame}
\frametitle{A cosa servono gli algoritmi}
\begin{block}{}
\begin{itemize}
\item L'identificazione di un algoritmo è il requisito indispensabile per risolvere un problema con il computer;
\item la scrittura di un problema con il computer consiste,in genere, nella traduzione di un algoritmo in qualche
\textCl{linguaggio di programmazione};
\end{itemize}
\end{block}
\pause
\begin{block}{}
\begin{center}
\large{\textbf{Prima di scrivere un programma è necessario individuare un algoritmo}}
\end{center}
\end{block}
\end{frame}

\section{Il linguaggio di programmazione JAVA}
\begin{frame}
\begin{block}{}
\begin{center}
\large{\textCl{Il linguaggio di programmazione JAVA}}
\end{center}
\end{block}
\end{frame}

\subsection{Il linguaggio JAVA}
\subsection{Un pò di storia}
\begin{frame}
\begin{block}{}
\begin{itemize}
\item 1954-1957 nasce il primo linguaggio di programmazione: \itshape{FORTRAN}
\item 1959 COBOL dove la B sta per Business. Infatti divenne uno dei primi linguaggi di 
programmazione orientato per le applicazioni business;
\item 1972 Dennis Ritchie fonda il linguaggio di programmazione C. E' molto potente come linguaggio di 
programmazione. 
\item Bjarne Strousturp sviluppo il C++ differente dal suo predecessore C++ è uno dei primi linguaggi
\textit{orientato a gli oggetti} che rappresenta un grande passo in avanti.
\item 1995 Sun Microsystem rilascia la prima versione ufficiale di JAVA. Aggiunge al C++ il concetto
\textit{Write Once, Run Anywhere}.
\end{itemize}
\end{block}
\end{frame}

\subsection{Programmazione orientata ad oggetti (OOP)}
\begin{frame}
\begin{block}{}
Java è un linguaggio orientato agli oggetti. Cosa vuol dire?
\end{block}
\begin{itemize}
\item In un linguaggio orientato agli oggetti puoi organizzare il tuo lavoro in \textCl{Oggetti} e \textCl{Classi}.
\end{itemize}
Esempio: immaginiamo di scrivere un programma che tiene traccia delle case di un nuovo condominio che si sta costruendo.
\begin{block}{}
Ogni casa è differente essenzialmente dalle altre per piccoli accorgimenti ad es.: per il suo interno, colore delle pareti, stile della cucina, tipo di bagno. Con JAVA ogni casa è un \textbf{\alert{OGGETTO}}.
\end{block}
\begin{block}{}
Sebbene le case differiscono leggermente una dall'altra la lista delle caratteristiche è sempre la stessa. Quindi 
all'interno del programma orientato agli oggetti ci sarà questa lista che contiene tutte le caratteristiche
della casa. La lista è chiamata \textbf{\alert{CLASSE}}
\end{block}
\end{frame}

\subsection*{Cosa c'è di buono nella programmazione ad oggetti}
\begin{frame}
\begin{block}{}
\begin{center}
\itshape{Quindi esiste una reale relazione tra classi ed oggetti. Il programmatore definisce una classe, e dalla definizione
delle classi, il computer crea oggetti individuali.}
\end{center}
\end{block}
\pause
\begin{block}{}
Supponiamo ora che il progetto cambi! Che la casa passa da un piano a due piani. E che le stanze al secondo piano per metà sono 
a tre e quattro stanze da letto.
\end{block}
\pause
\begin{block}{}
Bisogna buttare via tutto?
\end{block}
\end{frame}